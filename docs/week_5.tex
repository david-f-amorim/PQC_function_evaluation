\documentclass{beamer}

\usepackage{amsmath}
\usepackage{amssymb}
\usepackage{amsthm}
\usepackage{mathtools}
\usepackage[UKenglish]{babel}
\usepackage{enumerate}
\usepackage{graphicx}
\usepackage{braket}
\usepackage{esint}
\usepackage{float}
\usepackage{tabularx}
\usepackage{array}
\usepackage{subcaption}
\usepackage{hyperref}
\hypersetup{colorlinks=false, bookmarks=true}

\usetheme{Madrid}
\usecolortheme{seahorse}
\usefonttheme{professionalfonts}
\useinnertheme{circles}

\AtBeginSection[]
{
  \begin{frame}
    \frametitle{Table of Contents}
    \tableofcontents[currentsection]
  \end{frame}
}

\setbeamertemplate{caption}[numbered]

\title[QCNN State Preparation]{A QCNN for Quantum State Preparation}
\subtitle{Carnegie Vacation Scholarship}
\author[David Amorim]{David Amorim}
\institute[]{}
\date[05/08/2024]{Week 5 \\(29/07/2024 - 02/08/2024)}

\begin{document}

\frame{\titlepage}

\section{Preliminaries}
\begin{frame}
\frametitle{Aims for the Week}
The following aims were set at the last meeting (29/07/2024):

\begin{alertblock}{Improve Loss Function}
Work on an improved version of WILL. Incorporate some phase extraction metrics (e.g. $\chi$, $\epsilon$) into the loss function. 
\end{alertblock}

\begin{alertblock}{Investigate Phase Extraction}
Study the relationship between mismatch and the extracted phase, i.e. study the operator $\tilde{Q}^\dagger (\hat{I} \otimes\hat{R}) \tilde{Q}$. 
\end{alertblock}

\begin{alertblock}{Mitigate Barren Plateaus}
Work on strategies to mitigate barren plateaus, e.g. implement layer-by-layer training.
\end{alertblock}
\end{frame}

\begin{frame}
THIS WEEK INCLUDE EXAMPLES WITH HIGH $m$ !!
\end{frame}

\section{Next Steps}

\begin{frame}
\frametitle{Next Steps}
\begin{itemize}
\item ...
\end{itemize}
\end{frame}



\end{document}