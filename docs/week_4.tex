\documentclass{beamer}

\usepackage{amsmath}
\usepackage{amssymb}
\usepackage{amsthm}
\usepackage[UKenglish]{babel}
\usepackage{enumerate}
\usepackage{graphicx}
\usepackage{braket}
\usepackage{esint}
\usepackage{float}
\usepackage{subcaption}
\usepackage{hyperref}
\hypersetup{colorlinks=false, bookmarks=true}

\usetheme{Madrid}
\usecolortheme{seahorse}
\usefonttheme{professionalfonts}
\useinnertheme{circles}

\AtBeginSection[]
{
  \begin{frame}
    \frametitle{Table of Contents}
    \tableofcontents[currentsection]
  \end{frame}
}

\setbeamertemplate{caption}[numbered]

\title[PQC Function Evaluation]{PQC Function Evaluation}
\subtitle{Carnegie Vacation Scholarship}
\author[David Amorim]{David Amorim}
\institute[]{}
\date[29/07/2024]{Week 4 \\(22/07/2024 - 26/07/2024)}

\begin{document}

\frame{\titlepage}

\section{Aims}

\begin{frame}
\frametitle{Aims}
\begin{itemize}
\item Continue work on WIM (develop a distance measure taking into account $\Psi$, e.g. $w_j \sim L_\Psi (x,y)=\sum_j 2^{j-m} |x_j -y_j|$) 
\item Change IL structure (all-to-all...)
\item Keep parameters fixed for each layer
\end{itemize}
\end{frame}

\section{Keeping Parameters Fixed}

\begin{frame}
\frametitle{Keeping Parameters Fixed}
\begin{itemize}
\item Implemented the option to \alert{fix parameters for each layer type}
\item This means that each IL, neighbour-to-neighbour CL, and all-to-all CL use the same set of parameters 
\item This reduces the overall number of parameters to $n + m + \frac{1}{2} m * (m-1)$
\end{itemize}

ADDED FUNCTIONALITY, TEST !!! (does it make a difference?)
-- definitely works in terms of parameters 
-- not sure yet if makes a difference w.r.t. mismatch MAKES PERFORMANCE WORSE 
-- should at least speed things up? but apparently not ... DOES NOT SPEED THINGS UP...
\end{frame}

\section{Improving IL Structure}

\begin{frame}
\frametitle{Improving IL Structure}
-- added shifts to successive ILs (test effects..): seems to give slight improvements! 
-- added functionality for AA layers (test later)..
\end{frame}

\section{Improving the Loss Function}

\subsection{Preliminary Definitions}

\begin{frame}
\frametitle{Preliminary Definitions}
\begin{itemize}
\item Consider a \alert{computational basis state}, $\ket{j}$, in a $p$-qubit register:
\begin{equation}
\ket{j} = \bigotimes^{p-1}_{\alpha=0} \ket{j_\alpha}, \; \; \; \; \ket{j_\alpha} \in \{\ket{0},\ket{1} \}
\end{equation}
\item Define 
\begin{equation}
j_\alpha \equiv \begin{cases}
0 & \text{if } \ket{j_\alpha}=\ket{0} \\
1 & \text{if }  \ket{j_\alpha}=\ket{1} \\
\end{cases}
\end{equation}
\item Two \alert{digitally encoded binary numbers} can be associated with $\ket{j}$:
\begin{align}
j &\equiv \sum_{\alpha=0}^{p-1} j_\alpha 2^{\alpha} &(0 \leq j \leq 2^p-1 ), \\
j' &\equiv \sum_{\alpha=0}^{p-1} j_\alpha 2^{\alpha-p} &(0 \leq j' \leq 1)
\end{align}
\end{itemize}
\end{frame}

\begin{frame}
\frametitle{Preliminary Definitions}
\begin{itemize}
\item Consider an $n$-qubit \alert{input register} and an $m$-qubit \alert{target register}, denoted with subscripts $i$ and $t$, respectively 
\item A computational basis state of the combined system, $\ket{k}_{i+t}$, can be decomposed into
\begin{equation}
\ket{k}_{i+t} = \ket{j}_i \otimes \ket{l}_t,
\end{equation}
for computational basis states $\ket{j}_i$, $\ket{l}_t$ of the two registers
\item Define 
\begin{align}
\mathsf{input}(\ket{k}_{i+t}) &\equiv \ket{j}_i \\
\mathsf{target}(\ket{k}_{i+t}) &\equiv \ket{l}_t \\
\end{align}
\item A \alert{general state of the two-register system} is then 
\begin{equation}
\ket{z} = \sum^{2^{n+m}-1}_{k=0} z_k \ket{k}_{i+t}
\end{equation}
\end{itemize}
\end{frame}

\begin{frame}
\frametitle{Preliminary Definitions}
\begin{itemize}
\item For training in superposition, the two-register target state is 
\begin{equation}
\ket{y} = \sum_{j=0}^{2^n -1} \frac{1}{\sqrt{2^n}} \ket{j}_i \ket{\Psi'(j)}_t \equiv \sum_{k=0}^{2^{n+m}-1} y_k \ket{k}_{i+t}, 
\end{equation}
with $y_k =0$ if target($\ket{k}_{i+t}$)
\end{itemize}
\end{frame}

\begin{frame}
\frametitle{Preliminary Definitions}
\begin{itemize}
\item FIND GOOD NOTATION FOR ALL THIS !!!
\item THINK ABOUT THIS ALL MORE DEEPLY! MAYBE HAVE TO SWITCH TO WEIGHTED L1/2 LOSS INSTEAD? (WILL)
\end{itemize}
\end{frame}

\begin{frame}
\frametitle{Improving WIM}
\begin{itemize}
\item Recall the definition of \alert{WIM} (\alert{W}e\alert{I}ghted \alert{M}ismatch):
\begin{equation}
\mathsf{WIM}(x,y) =  \left\vert 1 - \sum_{k=0}^{2^{n+m}-1} \tilde{w}_k x_k y_k \right \vert, 
\end{equation}
where
\begin{itemize}
\item $\ket{x} = \sum_{k=0}^{2^{n+m}-1} x_k \ket{k} $ is the output state produced by the QCNN,
\item $\ket{y} = \sum_{k=0}^{2^{n+m}-1} y_k \ket{k} $ is the target state, 
\item $\tilde{w}_k \in \mathbb{R}_+$ are weighting factors 
\end{itemize}
\item .... calculate $w_k$
\item $\tilde{w}$
\end{itemize}
\end{frame}





\end{document}