\documentclass{beamer}

\usepackage{amsmath}
\usepackage{amssymb}
\usepackage{amsthm}
\usepackage{mathtools}
\usepackage[UKenglish]{babel}
\usepackage{enumerate}
\usepackage{graphicx}
\usepackage{braket}
\usepackage{esint}
\usepackage{float}
\usepackage{tabularx}
\usepackage{array}
\usepackage{subcaption}
\usepackage{hyperref}
\usepackage{xcolor}
\hypersetup{colorlinks=false, bookmarks=true}
\usepackage{tikz}
\usetikzlibrary{quantikz2}
\usepackage{adjustbox}



\usetheme{Madrid}
\usecolortheme{seahorse}
\usefonttheme{professionalfonts}
\useinnertheme{circles}

\AtBeginSection[]
{
  \begin{frame}
    \frametitle{Table of Contents}
    \tableofcontents[currentsection]
  \end{frame}
}

\setbeamertemplate{caption}[numbered]

\title[QCNN State Preparation]{A QCNN for Quantum State Preparation}
\subtitle{Carnegie Vacation Scholarship}
\author[David Amorim]{David Amorim}
\institute[]{}
\date[21/08/2024]{Weeks 7-8 \\(12/08/2024 - 23/08/2024)}

\begin{document}

\frame{\titlepage}

\begin{frame}
\frametitle{Aims for the Week}
The following aims were set at the last meeting (14/08/2024):

\begin{alertblock}{New Phase Encoding Approach}
Investigate a new approach to phase encoding using linear piecewise phase functions without explicit function evaluation.  
\end{alertblock}

\begin{alertblock}{Handover}
Hand over the slides, documentation, code and the poster for the Carnegie Trust.
\end{alertblock}
\end{frame}

\section{Phase Encoding}

\begin{frame}
\frametitle{Preliminaries}

\begin{itemize}
\item Consider an \alert{$n$-qubit register} with computational basis states $\ket{j} = \ket{j_0 j_1 ... j_{n-1}}$ representing $n$-bit strings
\item Let \alert{$p$} of the register qubits be \alert{precision qubits} so that 
\begin{equation} 
j = \sum^{n -1}_{k=0} j_k 2^{k-p}
\end{equation}
\item Now consider a \alert{phase function} $\Psi$ over the domain $\mathcal{D} = \{ j \}$ and construct an \alert{$M$-fold partition} into sub-domains $\mathcal{D}_u$:
\begin{equation}
\mathcal{D} = \bigcup_{u=1}^M \mathcal{D}_u, \; \; \; \mathcal{D}_u \cap \mathcal{D}_v = \emptyset, 
\end{equation}
\item Take \alert{$M = 2^m$} with $m \leq n$ and let the sub-domains be equally sized $(|\mathcal{D}_u| = |\mathcal{D}_v|$)
\end{itemize}
\end{frame}

\begin{frame}
\frametitle{Preliminaries}
\begin{alertblock}{Aim}
Construct an appropriate operator to transform 
\begin{equation}
\ket{j} \mapsto e^{i \Psi (j)} \ket{j}
\end{equation}
via the linear piecewise approximation
\begin{equation}
\ket{j} \mapsto e^{i (\alpha_u j + \beta_u)} \ket{j} \; \; \; (j \in \mathcal{D}_u)
\end{equation}
\end{alertblock}
\end{frame}

\begin{frame}
\frametitle{Initial Remarks}
\begin{itemize}
\item The $2^m$ pairs of coefficients $(\alpha_u, \beta_u)$ require \alert{$2^m$ independent operators} $\hat{O}_u$ to implement the mapping $\ket{j} \mapsto e^{i (\alpha_u j + \beta_u)} \ket{j}$
\item Each operator $\hat{O}_u$ will generally involve \alert{controlled rotations} on all \alert{$n$ qubits} in the register, with \alert{$m$} qubits acting as \alert{controls}  
\item Thus, the expected lower bound for controlled rotations is \alert{$\sim \Omega (2^m n)$}
\item Note that \alert{$m$-controlled operations} require \alert{$\Theta(m^2)$} CNOT gates [\emph{Barenco 1995}\footnote{\url{https://arxiv.org/pdf/quant-ph/9503016}}, Cor 7.6] or \alert{$\Theta(m)$} CNOT gates when using ancillae [\emph{Barenco 1995}, Cor 7.12]
\item To avoid this additional factor in the gate count and meet the lower bound \alert{only single-controlled operations} will be employed, leading to a more complex control architecture 
\end{itemize}
\end{frame}

\begin{frame}
\frametitle{Constructing $\hat{O}_u$}
\begin{itemize}
\item Consider the single-qubit operators 
\begin{equation}
\hat{P}^{(k)}(\varphi) = \begin{pmatrix}
e^{i \varphi} & 0 \\ 0 & e^{i \varphi}
\end{pmatrix}, \; \; \; \hat{R}^{(k)}(\varphi) = \begin{pmatrix}
1 & 0 \\ 0 & e^{i \varphi}
\end{pmatrix}
\end{equation}
each acting on the $k$th qubit
\item Now define \alert{
\begin{equation}
\hat{U}^{(k)}_u \equiv \hat{P}^{(k)} (\beta_u /n ) \hat{R}^{(k)} (\alpha_u 2^{k-p})
\end{equation}}
\item Then \alert{
\begin{equation}
\hat{O}_u \equiv \bigotimes^{n-1}_{k=0} \hat{U}^{(k)}_u
\end{equation}}
transforms 
\begin{equation}
\ket{j} \mapsto \exp \left[ i \left( \sum_{k=0}^{n-1} \alpha_u j_k 2^{k-p} + \beta_u \right) \right] \ket{j} = e^{i (\alpha_u j + \beta_u )} \ket{j}
\end{equation}
\end{itemize}
\end{frame}

\begin{frame}
\frametitle{The Control Structure}
\begin{itemize}
\item It is straight-forward to construct $\hat{O}_u$ for each of the sub-domains $\mathcal{D}_u$ 
\item More challenging is \alert{applying the correct $\hat{O}_u$} based on the sub-domain corresponding to each $\ket{j}$, which requires \alert{controlling} on the first \alert{$m$ qubits}
\item In order to achieve this with only \alert{single-controlled operations} a control structure similar to \emph{Barenco 1995} Lemmas 6.1, 7.1 is chosen   
\item This involves defining $2^m$ \alert{auxiliary operators $\hat{V}^{(k)}_q$} which give the $\hat{U}^{(k)}_u$ when multiplied in appropriate combinations
\item Since a product of rotation operators corresponds to a sum of rotation angles, the $\hat{V}^{(k)}_q$ can be constructed by solving the appropriate \alert{linear system} in the $\hat{U}^{(k)}_u$
\item The following two slides show examples of the control structure for \alert{`target qubits'}, i.e. the $n-m$ qubits that do not act as controls
\end{itemize}
\end{frame}

\begin{frame}
\frametitle{The Case $m=2$ ($M=4$)}
\begin{figure}
\begin{quantikz}[row sep={0.7cm,between origins}]
\lstick{$\ket{j_0}$} & & & \ctrl{1} &  & \ctrl{1} & \ctrl{2} & \rstick[2]{controls}  \\
\lstick{$\ket{j_1}$} & & \ctrl{1} & \targ{} & \ctrl{1} & \targ & & & \\
\lstick{$\ket{j_k}$} & \gate{V_0} & \gate{V_1} & & \gate{V_2} & & \gate{V_3} & \rstick{target} \\
\end{quantikz}
\caption{Control structure for $m=2$ ($M=4$) with $2 \leq k < n$. The number of controlled operations is $2^{m+1}-3=5$}
\end{figure}

\begin{table}
\centering 
\begin{tabular}{c | c | c || c | c | c }
$(j_0 j_1)$ & Operation & Equiv. $\hat{U}$ & $(j_0 j_1)$ & Operation & Equiv. $\hat{U}$ \\ \hline 
(00) & $\hat{V}_0$ & $\hat{U}_0$ & (10) & $\hat{V}_3 \hat{V}_2 \hat{V}_0 $ & $\hat{U}_2$ \\
(01) & $\hat{V}_2 \hat{V}_1 \hat{V}_0$ & $\hat{U}_1$ & (11) & $\hat{V}_3 \hat{V}_1 \hat{V}_0$ & $\hat{U}_3$ 
\end{tabular}
\caption{Operations applied to $\ket{j_k}$ for various control states}
\end{table}
\hspace{0.5cm}
\end{frame}

\begin{frame}
\frametitle{The Case $m=3$ ($M=8$)}
\begin{figure}
\begin{adjustbox}{width=\textwidth}
\begin{quantikz}[row sep={0.7cm,between origins}]
\lstick{$\ket{j_0}$} & &  \ctrl{3} & \ctrl{1} & & \ctrl{1} & & & & \ctrl{2} & & & & \ctrl{2} & &   \rstick[3]{controls} \\
\lstick{$\ket{j_1}$} &  & &\targ{} & \ctrl{2} & \targ{} & \ctrl{2}& \ctrl{1}&&&& \ctrl{1}&&  & &  \\
\lstick{$\ket{j_2}$} &  & & &&&& \targ{}& \ctrl{1}& \targ{}& \ctrl{1}& \targ{}& \ctrl{1} & \targ{} & \ctrl{1} &  \\
\lstick{$\ket{j_k}$} &  \gate{V_0} & \gate{V_1} & & \gate{V_2}& & \gate{V_3}& & \gate{V_4}& & \gate{V_5}&& \gate{V_6}&& \gate{V_7} &  \rstick[1]{target} \\
\end{quantikz}
\end{adjustbox}
\caption{Control structure for $m=3$ ($M=8$) with $3 \leq k < n$. The number of controlled operations is $2^{m+1}-3=13$}
\end{figure}

\begin{table}
\centering 
\begin{tabular}{c | c | c || c | c | c }
$(j_0 j_1 j_2)$ & Operation & Equiv. $\hat{U}$ & $(j_0 j_1 j_2)$ & Operation & Equiv. $\hat{U}$ \\ \hline 
(000) & $\hat{V}_0$ & $\hat{U}_0$ & (100) & $\hat{V}_6 \hat{V}_5 \hat{V}_2 \hat{V}_1 \hat{V}_0$ & $\hat{U}_4$ \\
(001) & $\hat{V}_7 \hat{V}_4 \hat{V}_0$ & $\hat{U}_1$ & (101) & $\hat{V}_7 \hat{V}_4 \hat{V}_2 \hat{V}_1 \hat{V}_0$ & $\hat{U}_5$ \\
(010) & $\hat{V}_5 \hat{V}_4 \hat{V}_2 \hat{V}_0$ & $\hat{U}_2 $& (110) & $\hat{V}_6 \hat{V}_4 \hat{V}_3 \hat{V}_1 \hat{V}_0  $ & $\hat{U}_6$ \\
(011) & $\hat{V}_7 \hat{V}_6 \hat{V}_3 \hat{V}_2 \hat{V}_0 $ & $\hat{U}_3$ & (111) & $\hat{V}_7 \hat{V}_5 \hat{V}_3 \hat{V}_1 \hat{V}_0$ & $\hat{U}_7$
\end{tabular}
\caption{Operations applied to $\ket{j_k}$ for various control states}
\end{table}
\end{frame}

\begin{frame}
\frametitle{The Control Structure}
\begin{itemize}
\item The control structure required to apply the appropriate $\hat{U}^{(k)}_u$ to the $k$-th target qubit requires \alert{$2^{m+1} -3$ CNOT gates}  
\item As there are $n-m$ target qubits this brings the CNOT count due to the targets to \alert{$(n-m) (2^{m+1}-3)$} 
\item Handling the control structure for the $m$ \alert{`control qubits'} requires slightly more care as the operator to be applied to the $l$-th control qubit is conditional on $\ket{j_l}$ itself 
\item This problem can be addressed by introducing an \alert{ancilla $\ket{0}_a$} and following the procedure:
\begin{enumerate}[(a)]
\item Apply a CNOT gate to the ancilla, controlled by $\ket{j_l}$
\item Apply the same control structure as for the target qubits, with the ancilla as the target 
\item Apply a SWAP gate between the ancilla and $\ket{j_l}$ 
\item Apply a CNOT gate to the ancilla, controlled by $\ket{j_l}$
\end{enumerate}
\item The final step clears the ancilla, allowing it to be \alert{re-used for all $m$} controls
\end{itemize}
\end{frame}

\begin{frame}
\frametitle{Gate Cost}
\begin{itemize}
\item Thus, encoding the phase on each control qubit requires the same structure as before but with an \alert{additional 5 CNOT} gates per control qubit (3 of are part of the SWAP)
\item The $m$ control qubits thus require \alert{$m 2^{m+2}$ CNOT} gates in addition to the \alert{$(n-m) (2^{m+1} -3)$ CNOTs} for the targets
\end{itemize}
\begin{alertblock}{Overall Complexity}
The CNOT cost of the algorithm presented here is 
\begin{equation}
C(n,m) = 2^{m+1} ( n+m) - 3 (n-m), 
\end{equation}
corresponding to the lower bound $\mathcal{O}(n2^m)$ on the complexity
\end{alertblock}
\end{frame}

\begin{frame}
\frametitle{Additional Remarks}
\begin{itemize}
\item Generally, when applying a controlled phase gate, the resulting phase shift cannot be unambiguously attributed to either the control or the target 
\item Here, this \alert{does not pose an issue} as only the overall phase of the $n$-qubit register matters 
\item \emph{Barenco 1995} \alert{omits the explicit construction} of the control structure for general $m$, only pointing towards the generalisation of the cases $m=2,3$ shown here 
\item Circuit structure may be \alert{simplified by using SWAP gates} and the ancilla for all $n$ qubits, at the cost of incurring $6(n-m)$ additional CNOT gates    
\end{itemize}
\end{frame}

\begin{frame}
\frametitle{Comparison with the Previous Approach}
\begin{itemize}
\item The phase encoding in \emph{Hayes 2023}\footnote{\url{https://arxiv.org/pdf/2306.11073}} uses $n_l$ \alert{label qubits} (with $2^{n_l} =M$) as well as $n_c$ \alert{coefficient qubits} 
\item The overall gate cost has contributions from the label operation ($\mathcal{O}(2^{n_l}n)$), the addition and multiplication operations ($\mathcal{O}(n^2 + n_c^2)$), as well as loading the coefficients ($\mathcal{O}(n_c 2^{n_l} n_l^2)$):
\begin{equation}
C_\text{Hayes}(n, n_c, n_l) = \mathcal{O}(n^2 + 2^{n_l} [n + n_l^2 n_c] + n_c^2)
\end{equation}
\begin{alertblock}{Complexity Comparison}
The new approach results in a \alert{quadratic complexity reduction} in $n$, from $\mathcal{O}(n^2)$ to $\mathcal{O}(n)$. The $\mathcal{O}(Mn)$ term remains while the number of ancillae is reduced from $n_c + n_l$ to 1 
\end{alertblock}
\end{itemize}

\end{frame}


\section{Handover}

\begin{frame}

\frametitle{Handover}

The code, documentation, slides, and poster are all available on GitHub:
\begin{center}
\vspace{0.6cm}
\href{https://github.com/david-f-amorim/PQC_function_evaluation}{\texttt{https://github.com/david-f-amorim/PQC\_function\_evaluation}}
\vspace{0.6cm}
\end{center}

\begin{itemize}
\item The source code is found in the directory \alert{\texttt{pqcprep}} 
\item The slides and poster are found in the directory \alert{\texttt{slides}}
\item The documentation is hosted externally  \href{https://david-f-amorim.github.io/PQC_function_evaluation/pqcprep.html}{\textcolor{purple}{here}}, which is also linked on GitHub
\end{itemize}

\end{frame}

\end{document}